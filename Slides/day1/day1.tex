\documentclass{beamer}
	\usefonttheme{default}     % Font theme: serif
	\usecolortheme[snowy]{owl}

\usepackage{parskip}
	\setlength{\parskip}{\smallskipamount} 

\usepackage{listings}
\usepackage{tikz}

\title{
\tiny{add some nice picture and/or structures, uni logos} \\
\vspace{0.3cm}
\Large{Topological Data Analysis in Python}}
\date{26\textsuperscript{th} - 28\textsuperscript{th} of October 2020}
\author[Michael Bleher]{organized by: \\  Michael Bleher, Maximilian Schmahl and Daniel Spitz}
\institute{Heidelberg University}

\begin{document}

\frame[plain]{
\tikz[remember picture,overlay]{
	\path (0,0) -- (0.5,0) node {\includegraphics[width=2cm,height=2cm,keepaspectratio]{../../Offizielles/STRUCTURES_bunt.png}};
	\path (0,0) -- (2,0) node {\includegraphics[width=2cm,height=2cm,keepaspectratio]{../../Offizielles/hd_logo_standard_16cm_rgb.png}};
	}
\titlepage
}

\AtBeginSection[]
{
	\begin{frame}{Contents}
	\setcounter{tocdepth}{1}
	\tableofcontents[currentsection, ]
	\end{frame}
}

\AtBeginSubsection[] {
    \begin{frame}<beamer>
    \frametitle{Inhalt} %
		\setcounter{tocdepth}{2}
    \tableofcontents[currentsubsection]
    \end{frame}
}


\section*{Contents}
\begin{frame}[plain]{Contents}
\setcounter{tocdepth}{1}
\tableofcontents
\end{frame}




\section{Programme}

\begin{frame}{Programme}
?
\end{frame}


\begin{frame}{scikit-tda}
scikit-tda Libraries
\begin{itemize}
	\item Ripser.py
  \item Kepler Mapper
  \item Persim
  \item CechMate
  \item TaDAsets
\end{itemize}
\end{frame}


\begin{frame}
copy scikit-tda slide, but insert schedule and highlight \& encircle the parts and when we will cover them
\end{frame}




\section{Topology in Data Analysis}

\begin{frame}{Topology}
Topology is the field of mathematics that studies shapes.

Topology is ``blind'' to continuous deformations.

add obligatory pictures and examples that give intuition of homotopy equivalence (interval, circle, sphere, punctured sphere, torus, coffee cup)
\end{frame}


\begin{frame}{Königsberg Problem}
``counting holes'' as a way to distinguish topological spaces
$V-E+F = 2$
introduce Graphs for later convenience
\end{frame}


\begin{frame}{Data}
Data = points in $\mathbb{R}^n$

Challenges of data science: size, complexity, curse of dimensionality
\end{frame}


\begin{frame}{Topology in Data?}
points are not continuous!

$\mathbb{R}^n$ has no interesting topology!

$\rightarrow$ where is the topology?
\end{frame}


\begin{frame}{Topology in Data}
pictures like the ones from carlsson's talk 

(regression, clusters, loops, bifurcations)
\end{frame}


\begin{frame}{Topology in Data}
noisy $S^1$, noisy pictures from MNIST or house numbers

$\rightarrow$ would like to have ways to extract these topological properties from the data
\end{frame}


\begin{frame}{Topological Data Analysis}
\begin{itemize}
	\item A lossy compressed mathematical representation of a data set. You can study the global structure of a dataset, down to the details of a single data point, without incurring a cognitive overload.
	\item Resistance to noise and missing data. TDA retains significant features of the data.
	\item Invariance. Only connectedness matters. The skew, size, or orientation of data does not fundamentally change that data.
	\item A data exploration tool. Get answers to questions you haven’t even asked yet.
  \item A methodology to study the shape of data and manifolds. TDA has a solid theoretical foundation and inherits functoriality.
\end{itemize}
\end{frame}




\section{The Mapper Algorithm}

\begin{frame}{Intro}
Recall pictures from ``The Shape of Data''
\end{frame}


\begin{frame}{Königsberg Problem (revisited)}
sth. about traveling salesman, distances, ...

such that the setup of Königsberg corresponds to a point cloud where the mapper network is interesting?

idea: prepare people for mapper with the following analogy

Bridges = ways to go from one part of the city to another
Edges = overlap
\end{frame}


\begin{frame}{Mapper}
The Mapper Algorithm (show items iteratively)
\begin{enumerate}
	\item Project(Filter dependency!)
	\item Cover (Cover dependency!)
	\item Clustering (metric dependeny!)
	\item Graph 
	\begin{itemize}
		\item For each \textbf{cluster} draw a \textbf{node}
		\item Whenever clusters \textbf{interesect} draw an \textbf{edge}
	\end{itemize}
	\item Prettify and Analyze
	%\begin{itemize}
		%\item Size the graph nodes by a function of interest (for instance: number of members inside cluster)
		%\item Color the graph nodes by a function of interest (for instance: average customer spend)
		%\item Shape the graph nodes by a function of interest (for instance: squares for avg_timestamp < 2015, circles for avg_timestamp >= 2015)
		%\item Size the graph edges by a function of interest (for instance: number of intersecting members between nodes)
		%\item Color the graph edges by a function of interest (for instance: average color of connected nodes)
		%\item Shape the graph edges by a function of interest (for instance: dotted line when intersecting members < 3 else solid line)
		%\item Provide descriptive statistics on the nodes and the graph. (for instance: Kolmogorov–Smirnov test between node variables and dataset variables.
	%\end{itemize}
\end{enumerate}
Pictures corresponding to the steps
\end{frame}


\begin{frame}{Examples}
some more examples to get a better feeling
also hint at or state exercises
\end{frame}


\section{Kepler Mapper}

\begin{frame}[fragile]{Kepler Mapper}
Test
\begin{lstlisting}[backgroundcolor = \color{bg!90!fg},
                   language = python,
                   xleftmargin = 1cm,
									]
	import kmapper
	
	data = some_way_to_load_data()
	
	kmapper(data, filter, bins)
\end{lstlisting}
\end{frame}


\begin{frame}{kmapper example}
first slide:
short look at Data

second slide top: code
second slide bottom: resulting graph

maybe several instances with different filter, cover, metric ...



\end{frame}






\end{document}